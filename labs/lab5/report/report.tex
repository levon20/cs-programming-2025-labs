\documentclass[]{vvsu}

\vvsuyear{2025}

%%%%%%%%%%%%%%%%%%%

\usepackage{graphicx} % для изображений
\usepackage{tabularray} % для таблиц
\usepackage{siunitx} % для обозначений (процент, градус)
\usepackage{listings} % для листингов кода

% Список путей, где будут искаться изображения и файлы
\graphicspath{{images/}}

% Автор документа
\author{Л.А. Сидоров}

% Настройка стилей для листингов кода
\input{listing_styles.tex}

%%%%%%%%%%%%%%%%%%%

\begin{document}

% Шапка
\vvsuhead{\linespread{1}\selectfont{}МИНОБРНАУКИ РОССИИ\\
\vspace{10pt}Федеральное государственное бюджетное образовательное учреждение\\
высшего образования\\
\fontsize{13}{13}\selectfont{}<<ВЛАДИВОСТОКСКИЙ ГОСУДАРСТВЕННЫЙ УНИВЕРСИТЕТ>>\\
(ФГБОУ ВО <<ВВГУ>>)\\
\vspace{10pt}\fontsize{12}{12}\selectfont{}ИНСТИТУТ ИНФОРМАЦИОННЫХ ТЕХНОЛОГИЙ И АНАЛИЗА ДАННЫХ\\
КАФЕДРА ИНФОРМАЦИОННЫХ ТЕХНОЛОГИЙ И СИСТЕМ}

% Название отчета
\title{Отчет\\по лабораторной работе №5}
\subtitle{по дисциплине\\<<Информатика и программирование>>}

% Участники работы
\member{Студент\\ гр. БИН-25-3}{Л.А. Сидоров}
\member{Ассистент\\ преподавателя}{М.В. Водяницкий}

% Вывод титульника
\maketitle

% Задание
\begin{addition}{Задание}
  Выполнить задания и оформить отчет по стандартам ВВГУ.

  \textit{\textbf{Задание 1.}}  
  Дан список из 10 различных целых чисел. Необходимо найти в нем число 3 и заменить на 30.

  \textit{\textbf{Задание 2.}}  
  Дан список из 5 целых чисел. Необходимо превратить его в список квадратов этих чисел.

  \textit{\textbf{Задание 3.}}  
  Имеется список различных целых чисел. Программа должна найти наибольшее из чисел списка и разделить его на длину списка.

  \textit{\textbf{Задание 4.}}  
  Имеется кортеж из нескольких произвольных элементов. Необходимо этот кортеж отсортировать. Если хотя бы один элемент не является числом, то кортеж остается неизменным.

  \textit{\textbf{Задание 5.}}  
  Имеется словарь товаров в магазине. Необходимо найти товар с минимальной и максимальной ценой.

  \textit{\textbf{Задание 6.}}  
  Имеется список произвольных элементов. Необходимо на основе этого списка создать словарь, где каждый элемент списка будет и ключом, и значением.

  \textit{\textbf{Задание 7.}}  
  Имеется словарь перевода английских слов на русский, где ключ английского слово, значение - русского. Необходимо реализовать программу которая получает на ввод русское слово и результатом выдает перевод на английский.

  \textit{\textbf{Задание 8.}}  
  Реализовать игру Камень-Ножницы-Бумага-Ящерица-Спок. Программа должна запрашивать у пользователя ввод одного из вариантов. Второй вариант случайно генерирует сама программа и возвращает победителя.

  Правила игры следующие:
  \begin{vvsu_itemize}
    \item Ножницы режут бумагу
    \item Бумага покрывает камень
    \item Камень давит ящерицу
    \item Ящерица отравляет Спока
    \item Спок ломает ножницы
    \item Ножницы обезглавливают ящерицу
    \item Ящерица съедает бумагу
    \item Бумага подставляет Спока
    \item Спок испаряет камень
    \item Камень разбивает ножницы
  \end{vvsu_itemize}

  \textit{\textbf{Задание 9.}}  
  Дан список слов - например:
  
  ["яблоко", "груша", "банан", "киви", "апельсин", "ананас"]

  Необходимо создать новый словарь, где:

  \begin{vvsu_itemize}
    \item Ключом будет первая буква слова
    \item Значением - список всех слов, начинающихся с этой буквы
  \end{vvsu_itemize}
  Пример результата:

  {'я': ['яблоко'], 'г': ['груша'], 'б': ['банан'], 'к': ['киви'], 'а': ['апельсин', 'ананас']}

  \textit{\textbf{Задание 10.}}  
  Дан список кортежей, где каждый кортеж содержит имя студента и его оценки, например:

  [("Анна", [5, 4, 5]), ("Иван", [3, 4, 4]), ("Мария", [5, 5, 5])]

  Необходимо:

  \begin{vvsu_itemize}
    \item Создать словарь, где ключ - имя студента, значение - его средняя оценка
    \item Найти студента с наибольшей средней оценкой и вывести его имя и средний балл
  \end{vvsu_itemize}
  
  Пример результата:

  Мария имеет наивысший средний балл: 5.0
\end{addition}


% Содержание
\toc

% Глава - Выполнение работы
\section{Выполнение работы}

% Подглава - Задание 1
\subsection{Задание 1}

Процедура на вход принимает указатель на первый элемент массива и его размер. Далее в цикле fot мы пробегаемся по элементам массива в поисках числа 3, если находим, то менаем значение на 30, в ином случае массив остается прежним. На рисунке \ref{fig:code_task_1} представлен код полученной программы.

\begin{vvsu_figure}{Листинг программы для задания 1}{fig:code_task_1}
  \begin{minipage}{.75\textwidth}
    \lstinputlisting[language=C,basicstyle=\fontsize{10}{10}\linespread{1}\selectfont\ttfamily]{code/task1.c}
  \end{minipage}
\end{vvsu_figure}

% Подглава - Задание 2
\subsection{Задание 2}

Процедура на вход принимает указатель на первый элемент массива и его размер. Далее мы пробегаемся с помощью цикла for по массиву и возводим все его элементы в квадрат.

\begin{vvsu_figure}{Листинг программы для задания 2}{fig:code_task_2}
  \begin{minipage}{.75\textwidth}
    \lstinputlisting[language=C,basicstyle=\fontsize{10}{10}\linespread{1}\selectfont\ttfamily]{code/task2.c}
  \end{minipage}
\end{vvsu_figure}

% Подглава - Задание 3
\subsection{Задание 3}

Функция на вход принимает указатель на первый элемент массива и его размер. Далее заводим переменные maxNum типа integer, инициализируемую нулевым элементом массива. Именно ее мы в цикле for, пробегая по массиву, будем сравнивать с его элементами и искать наибольшее значение. После нахождения функция возвращает значение, деленное на размер массива. На рисунке \ref{fig:code_task_3} представлен код программы.
\begin{vvsu_figure}{Листинг программы для задания 3}{fig:code_task_3}
  \begin{minipage}{.75\textwidth}
    \lstinputlisting[language=C,basicstyle=\fontsize{10}{10}\linespread{1}\selectfont\ttfamily]{code/task3.c}
  \end{minipage}
\end{vvsu_figure}

% Подглава - Задание 4
\subsection{Задание 4}

Функция на фход принимает ссылку на картеж. Далее в переменную size мы записываем его размер. После в цикле for пробегаемся по картежу, проверяя его элементы на тип данных, в случае, если находим не число, то возвращаем картеж без изменений. В ином случае извлекаем его элементы в список и сортируем. В конце конвертируем список в кортеж и возвращаем измененную версию. На рисунке \ref{fig:code_task_4} представлен код решения.

\begin{vvsu_figure}{Листинг программы для задания 4}{fig:code_task_4}
  \begin{minipage}{.75\textwidth}
    \lstinputlisting[language=Python,basicstyle=\fontsize{10}{10}\linespread{1}\selectfont\ttfamily]{code/task4.py}
  \end{minipage}
\end{vvsu_figure}


% Подглава - Задание 5
\subsection{Задание 5}

Функция принимает на вход словарь с прдуктами и ценами. Далее из этого словаря мы извлекаем в отдельные списки продукты и цены по соответствующим индексам. После находим в списке цен индексы с минимальной и максимальной ценами. В конце возвращаем словрь, подобный словарь, однако состоящий лишь из продуктов с минимаольной и максимальной ценами. На рисунке \ref{fig:code_task_5} представлен код программы.

\begin{vvsu_figure}{Листинг программы для задания 5}{fig:code_task_5}
  \begin{minipage}{.75\textwidth}
    \lstinputlisting[language=Python,basicstyle=\fontsize{10}{10}\linespread{1}\selectfont\ttfamily]{code/task5.py}
  \end{minipage}
\end{vvsu_figure}

% Подглава - Задание 6
\subsection{Задание 6}

Функция получает на вход список. В переменную size мы запишем его размер и создадим пока пустой словарь. Далее в цикле for пробежимся по списку и каждому элементу списка создадим в словаре объект с ключем и значением, соответствующими содержанию элемента списка. В конце она возвращает полученный словарь. На рисунке \ref{fig:code_task_6} представлен код программы.

\begin{vvsu_figure}{Листинг программы для задания 6}{fig:code_task_6}
  \begin{minipage}{.75\textwidth}
    \lstinputlisting[language=Python,basicstyle=\fontsize{10}{10}\linespread{1}\selectfont\ttfamily]{code/task6.py}
  \end{minipage}
\end{vvsu_figure}

% Подглава - Задание 7
\subsection{Задание 7}

Функция на вход принимает словарь из английских слов и их русским переводом. Далее из этого словаря мы извлекаем в отдельные списки английские и русские слова по соответствующим индексам. После пробегаемся по русскому списку в поисках указанного слова. Если находим, то возвращаем элемент из английского списка с нужным индексом. В ином случае возвращаем None. На рисунке \ref{fig:code_task_7} представлен код программы.

\begin{vvsu_figure}{Листинг программы для задания 7}{fig:code_task_7}
  \begin{minipage}{.75\textwidth}
    \lstinputlisting[language=Python,basicstyle=\fontsize{10}{10}\linespread{1}\selectfont\ttfamily]{code/task7.py}
  \end{minipage}
\end{vvsu_figure}

% Подглава - Задание 8
\subsection{Задание 8}

Функция на вход принимает ход игрока. Далее мы создаем список из возможных ходов и словарь с правилами, где указан ход в качестве ключа и то, при каких обстоятельствах он выйигрышный в качестве занчения. После генерируется ход программы. В конце мы проверяем, если ходы одинаковые, то возвращаем ничью, если ход программы содержится в значениях словаря под ходом игрока, то возвращаем, что игрок победил, в ином случае, возвращаем, что он проиграл. На рисунке \ref{fig:code_task_8} представлен код программы.

\begin{vvsu_figure}{Листинг программы для задания 8}{fig:code_task_8}
  \begin{minipage}{.75\textwidth}
    \lstinputlisting[language=Python,basicstyle=\fontsize{10}{10}\linespread{1}\selectfont\ttfamily]{code/task8.py}
  \end{minipage}
\end{vvsu_figure}

% Подглава - Задание 9
\subsection{Задание 9}

Функция на входи принимает список слов. Длаее мы создаем пустой словарь, в который будем записывать слово, в качестве значения и первую букву как ключ. После циклом for пробегаемся по списку и смотрим на каждый элемент, проверяем, есть ли он уже в словаре, если нет, то добавляем ключ и значение. В конце возвращаем полученный словарь. На рисунке \ref{fig:code_task_9} представлен код программы.

\begin{vvsu_figure}{Листинг программы для задания 9}{fig:code_task_9}
  \begin{minipage}{.75\textwidth}
    \lstinputlisting[language=Python,basicstyle=\fontsize{10}{10}\linespread{1}\selectfont\ttfamily]{code/task9.py}
  \end{minipage}
\end{vvsu_figure}

% Подглава - Задание 10
\subsection{Задание 10}

Функция на вход принимает список картежей с именами и оценками учеников. В цикле for мы пробегаемся по картежам списка, получаем имя и оценки ученика. Далее записываем имя в качестве ключа и средний бал в качестве занчения в созданный словарь. По пути программа ищет наивысший балл. В результате функция возвращает строку с именем и высшим балом ученика. 
На рисунке \ref{fig:code_task_10} представлен код программы.

\begin{vvsu_figure}{Листинг программы для задания 10}{fig:code_task_10}
  \begin{minipage}{.75\textwidth}
    \lstinputlisting[language=Python,basicstyle=\fontsize{10}{10}\linespread{1}\selectfont\ttfamily]{code/task10.py}
  \end{minipage}
\end{vvsu_figure}

Примечание: 1 - 3 решения представляют из себя отдельные процедуры, обрабатывающие и именающие полученныые данные. 4 - 10 решения представляют из себя отдельные функции, возращяющие то, что просят.

\end{document}