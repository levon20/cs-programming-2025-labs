\documentclass[]{vvsu}

\vvsuyear{2025}

%%%%%%%%%%%%%%%%%%%

\usepackage{graphicx} % для изображений
\usepackage{tabularray} % для таблиц
\usepackage{siunitx} % для обозначений (процент, градус)
\usepackage{listings} % для листингов кода

% Список путей, где будут искаться изображения и файлы
\graphicspath{{images/}}

% Автор документа
\author{Л.А. Сидоров}

% Настройка стилей для листингов кода
\input{listing_styles.tex}

%%%%%%%%%%%%%%%%%%%

\begin{document}

% Шапка
\vvsuhead{\linespread{1}\selectfont{}МИНОБРНАУКИ РОССИИ\\
\vspace{10pt}Федеральное государственное бюджетное образовательное учреждение\\
высшего образования\\
\fontsize{13}{13}\selectfont{}<<ВЛАДИВОСТОКСКИЙ ГОСУДАРСТВЕННЫЙ УНИВЕРСИТЕТ>>\\
(ФГБОУ ВО <<ВВГУ>>)\\
\vspace{10pt}\fontsize{12}{12}\selectfont{}ИНСТИТУТ ИНФОРМАЦИОННЫХ ТЕХНОЛОГИЙ И АНАЛИЗА ДАННЫХ\\
КАФЕДРА ИНФОРМАЦИОННЫХ ТЕХНОЛОГИЙ И СИСТЕМ}

% Название отчета
\title{Отчет\\по лабораторной работе №6}
\subtitle{по дисциплине\\<<Информатика и программирование>>}

% Участники работы
\member{Студент\\ гр. БИН-25-3}{Л.А. Сидоров}
\member{Ассистент\\ преподавателя}{М.В. Водяницкий}

% Вывод титульника
\maketitle

% Задание
\begin{addition}{Задание}
  Выполнить задания и оформить отчет по стандартам ВВГУ.

  \textit{\textbf{Задание 1.}}  
  Написать функцию, которая конвертирует время из одной величины в другую.

  На вход подается:

  \begin{vvsu_itemize}
  \item число (величина времени)
  \item исходная единица измерения
  \item единица измерения, в которую нужно перевести
  \end{vvsu_itemize}

  Функция должна вернуть конвертированное значение

  \textit{\textbf{Задание 2.}}  
  Пользователь делает вклад в банке в размере A рублей сроком на N лет

  Процент по вкладу зависит от суммы и срока

  Зависимость от суммы:
  \begin{vvsu_itemize}
  \item каждые 10 000 рублей увеличивают ставку на 0.3\%
  \item но суммарное увеличение не может превышать 5\%
  \item минимальный вклад - 30 000 рублей
  \end{vvsu_itemize}

  Зависимость от срока:
  \begin{vvsu_itemize}
  \item первые 3 года - 3\%
  \item от 4 до 6 лет - 5\%
  \item более 6 лет - 2\%
  \end{vvsu_itemize}

  Необходимо написать функцию, которая рассчитывает прибыль пользователя без учета первоначально вложенной суммы

  Используется сложный процент: каждый год процент начисляется на текущую сумму вклада

  На вход подаются: сумма вклада и количество лет. Результат: сумма прибыли (не весь вклад, а только заработанные проценты)

  \textit{\textbf{Задание 3.}}  
  Написать функцию для вывода всех простых чисел в заданном диапазоне. Нужно учитывать некорректные данные (например, начало больше конца или диапазон без простых чисел)

  На вход подаются два числа: начало и конец диапазона (включительно). На выходе - список всех простых чисел или сообщение об ошибке

  \textit{\textbf{Задание 4.}}  
  Реализовать функцию сложения двух матриц

  При сложении двух матриц получается новая матрица того же размера, где каждый элемент - это сумма элементов с тем же индексом из двух исходных матриц

  Ограничения:
  \begin{vvsu_itemize}
  \item складывать можно только матрицы одинакового размера
  \item размер матрицы должен быть строго больше 2 (например, 3×3, 4×4 и т.д.)
  \item при нарушении условий нужно вывести сообщение об ошибке
  \end{vvsu_itemize}

  На вход подаются:

  \begin{vvsu_itemize}
  \item размер матрицы n (для квадратной матрицы n × n)
  \item элементы первой матрицы (по строкам, через пробел)
  \item элементы второй матрицы в таком же формате
  \end{vvsu_itemize}

  Результат - новая матрица (в том же формате), либо сообщение об ошибке

  \textit{\textbf{Задание 5.}}  
  Написать функцию, которая определяет, является ли строка палиндромом

  Палиндром - это строка, которая читается одинаково слева направо и справа налево (обычно без учета пробелов, регистра и знаков препинания - эти правила нужно явно задать в своей реализации)

  На вход подается строка. На выходе:

  \begin{vvsu_itemize}
  \item Да, если это палиндром
  \item Нет, если это не палиндром
  \end{vvsu_itemize}

\end{addition}


% Содержание
\toc

% Глава - Выполнение работы
\section{Выполнение работы}

% Подглава - Задание 1
\subsection{Задание 1}

  Функция принимает на вход 3 параметра: time типа float, unit1 и unit2 типа char. Первым указывается время, вторым текущая единица измерения и третьм желаемая единица. Далее были заведены 3 константы с говорящими названиями для удобной конвертации. Следующим шагом обрабатываем входные данные на корректность, проверяем, что значение больше 0. 

\begin{vvsu_figure}{Листинг программы для задания 1}{fig:code_task_1}
  \begin{minipage}{.75\textwidth}
    \lstinputlisting[language=C,basicstyle=\fontsize{10}{10}\linespread{1}\selectfont\ttfamily]{code/task1.c}
  \end{minipage}
\end{vvsu_figure}

  Далее с помощью условий выясняем какая единица измерения дана и в какую из нее нужно перевести и переводим по соответствующим формулам. На рисунке \ref{fig:code_task_1} представлен код полученной программы.

% Подглава - Задание 2
\subsection{Задание 2}
  Функция на вход принимает 2 значения: startSum типа float и years типа int. Первым указывается вносимая сумма, а вторым - срок, на который хотим создать вклад. После проверяем данные на корректность: сумма должна быть больше 0 и кол-во лет тоже больше 0, в противном случае возвращаем 0. Далее заводим такой список переменных:
  \begin{vvsu_itemize}
  \item sum типа float, которая будет хранить накопленную сумму.
  \item rate типа float, которая будет хранить коэффициент на текущий период.
  \item bonusRate типа float, которая будет хранить бонусный коэффициент за каждые 10 тысяч вклада.
  \item firstRate типа float, которая будет хранить коэффициент для первых 3-ех лет вклада.
  \item seconRate типа float, которая будет хранить коэффициент для следующих 3-ех лет вклада.
  \item thirdRate типа float, которая будет хранить коэффициент для всех последующих лет вклада.
  \item maxBonusRate типа float, которая будет хранить максимальное значение бонусного коэффициента.
  \end{vvsu_itemize}

\begin{vvsu_figure}{Листинг программы для задания 2}{fig:code_task_2}
  \begin{minipage}{.75\textwidth}
    \lstinputlisting[language=C,basicstyle=\fontsize{10}{10}\linespread{1}\selectfont\ttfamily]{code/task2.c}
  \end{minipage}
\end{vvsu_figure}

 Следующим шагом считаем бонусный коэффициент и в случае, если он превышет макимальное значение, присваиваем максимальное значение. После по формуле: sum * (1 + rate) \^ years считаем суммы по соответствующим периодам. В конце от итоговой суммы отнимаем начальную и выводим значение. На рисунке \ref{fig:code_task_2} представлен код полученной программы.

% Подглава - Задание 3
\subsection{Задание 3}

  Процедура на вход принимает верхнюю и нижнюю границы диапазона, простые числа которого необходимо вывести. Параметры принимаются в переменные low типа int и high типа int соответственно. Первым делом заводим счетчик cntSimpleNums типа int для подсчета найденных простых чисел. Далее проверяем данные на корректность: нижняя граница не может быть больше верхней, а верхняя не может быть отрицательной. 

\begin{vvsu_figure}{Листинг программы для задания 3}{fig:code_task_3}
  \begin{minipage}{.75\textwidth}
    \lstinputlisting[language=C,basicstyle=\fontsize{10}{10}\linespread{1}\selectfont\ttfamily]{code/task3.c}
  \end{minipage}
\end{vvsu_figure}

  После проверяем, если нижняя граница < 3, то выведем заведомо простые числа, дабы не усложнять алгоритм, поскольку дальше, к примеру та же 2 вообще не будет существовать для программы, поскольку она четная, а за исключением ее, все остальные четные числа не простые. В конце проверяем осташвиеся числа на простоту и выводим найденные или сообщение о том, что в заданном диапазоне нет таких чисел. На рисунке \ref{fig:code_task_3} представлен код программы.

% Подглава - Задание 4
\subsection{Задание 4}

  Заводим 2 переменные типа int - numOfStr, что будет хранить кол-во строк, и numOfCol, что будет хранить кол-во столбцов. Далее просим пользователя ввести соответствующие параметры. После ввода проверяем их на корректность: оба должны быть больше 0. После чего объявляем 3 двумерных массива одинакового размера, соответствующего параметрам. matrix0 хранит первую матрицу, а matrix1 - вторую, а  resMatrix служит для результата сложения матриц. 

\begin{vvsu_figure}{Листинг программы для задания 4}{fig:code_task_4}
  \begin{minipage}{.75\textwidth}
    \lstinputlisting[language=C,basicstyle=\fontsize{10}{10}\linespread{1}\selectfont\ttfamily]{code/task4.c}
  \end{minipage}
\end{vvsu_figure}

  Затем просим пользователя ввести значения соответствующих матриц. А в конце с помощью вложенных циклов for складываем соответствующие элементы матрицы и записываем в соответствующую ячейку результирующей матрицы. Последник шагом выводим результат. На рисунке \ref{fig:code_task_4} представлен код решения.

% Подглава - Задание 5
\subsection{Задание 5}

  Функция на вход принимает строку и ее длинну. Далее заводим массив для новой строки, которая появится после удаления пробелов и переменную, в которой будет хранится индекс последнего символа новой строки. После удаляем все пробелы и запоминаем индекс последнего символа. 

\begin{vvsu_figure}{Листинг программы для задания 5}{fig:code_task_5}
  \begin{minipage}{.75\textwidth}
    \lstinputlisting[language=C,basicstyle=\fontsize{10}{10}\linespread{1}\selectfont\ttfamily]{code/task5.c}
  \end{minipage}
\end{vvsu_figure}

  Затем с помощью цикла for запускаем сравнение i-ого и indexOfEndStr-i символов строки. В сулчае неравенства выводим сообщение, что слово не палиндром, и завершаем программу.  На рисунке \ref{fig:code_task_5} представлен код программы.

\end{document}