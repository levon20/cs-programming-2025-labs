\documentclass[]{vvsu}

\vvsuyear{2025}

%%%%%%%%%%%%%%%%%%%

\usepackage{graphicx} % для изображений
\usepackage{tabularray} % для таблиц
\usepackage{siunitx} % для обозначений (процент, градус)
\usepackage{listings} % для листингов кода

% Список путей, где будут искаться изображения и файлы
\graphicspath{{images/}}

% Автор документа
\author{Л.А. Сидоров}

% Настройка стилей для листингов кода
\input{listing_styles.tex}

%%%%%%%%%%%%%%%%%%%

\begin{document}

% Шапка
\vvsuhead{\linespread{1}\selectfont{}МИНОБРНАУКИ РОССИИ\\
\vspace{10pt}Федеральное государственное бюджетное образовательное учреждение\\
высшего образования\\
\fontsize{13}{13}\selectfont{}<<ВЛАДИВОСТОКСКИЙ ГОСУДАРСТВЕННЫЙ УНИВЕРСИТЕТ>>\\
(ФГБОУ ВО <<ВВГУ>>)\\
\vspace{10pt}\fontsize{12}{12}\selectfont{}ИНСТИТУТ ИНФОРМАЦИОННЫХ ТЕХНОЛОГИЙ И АНАЛИЗА ДАННЫХ\\
КАФЕДРА ИНФОРМАЦИОННЫХ ТЕХНОЛОГИЙ И СИСТЕМ}

% Название отчета
\title{Отчет\\по лабораторной работе №4}
\subtitle{по дисциплине\\<<Информатика и программирование>>}

% Участники работы
\member{Студент\\ гр. БИН-25-3}{Л.А. Сидоров}
\member{Ассистент\\ преподавателя}{М.В. Водяницкий}

% Вывод титульника
\maketitle

% Задание
\begin{addition}{Задание}
  Выполнить задания и оформить отчет по стандартам ВВГУ.

  \textit{\textbf{Задание 1.}}  
  Написать программу, которая определяет, как будет вести себя кондиционер. Если температура в помещении 20 градусов и выше, то кондиционер выключается, если меньше - включается. Температура должна вводится пользователем с консоли.

  \begin{verbatim}
  Пример:
    Введите температуру: 18
    Кондиционер включен
  \end{verbatim}

  \textit{\textbf{Задание 2.}}  
  Год делится на четыре сезона: зима, весна, лето и осень. Написать программу, которая запрашивает у пользователя номер месяца и выводит к какому сезону этот месяц относится.

  \begin{verbatim}
  Пример:
    Введите номер месяца: 4
    Это весна 
  \end{verbatim}

  \textit{\textbf{Задание 3.}}  
  Считается, что один год, прожитый собакой, эквивалентен семи человеческим годам. При этом зачастую не учитывается, что собаки становятся абсолютно взрослыми уже к двум годам. Таким образом, многие предпочитают каждый из первых двух лет жизни собаки приравнивать к 10.5 годам человеческой жизни, а все последующие к 4.
  
  Написать программу, которая будет переводить собачий возраст в человеческий. Программа должна корректно обрабатывать входные данные и выводить соответствующие сообщения об ошибках:

  \begin{vvsu_itemize}
    \item Если вводится не число
    \item Если вводится число меньше 1
    \item Если вводится число большее 22
  \end{vvsu_itemize}

  \begin{verbatim}
  Пример:
    Введите возраст собаки (в годах): 5
    Возраст собаки в человеческих годах: 33.0
  \end{verbatim}

  \begin{verbatim}
  Пример:
    Введите возраст собаки (в годах): 0
    Ошибка: возраст должен быть не меньше 1
  \end{verbatim}

  \textit{\textbf{Задание 4.}}  
  Число делиться на 6 только в случае соблюдения двух условий:

  \begin{vvsu_itemize}
    \item Последняя цифра четная
    \item Сумма всех цифр делиться на 3
  \end{vvsu_itemize}

  Написать программу, которая выведет делиться ли введенное число на 6 или нет.

  \textit{\textbf{Задание 5.}}  
  Написать программу, которая будет проверять пароль на надежность. Пароль считается надежным, если его длина не менее 8 символов и если он содержит:

  \begin{vvsu_itemize}
    \item Заглавные буквы латиницы
    \item Строчные буквы латиницы
    \item Числа
    \item Специальные знаки
  \end{vvsu_itemize}

  В случае, если пароль не проходит по одному из условий, необходимо сообщить пользователю каким именно условиям он не удовлетворяет.

  \begin{verbatim}
  Пример:
    Введите пароль: qwerty
    Пароль ненадежный: отсутствуют заглавные буквы, числа и специальные символы
  \end{verbatim}

  \textit{\textbf{Задание 6.}}  
  Написать программу, которая определяет, является ли введенный пользователем год високосным. Год считается високосным, если он делится на 4, но не делится на 100, либо если он делится на 400.

  \begin{verbatim}
  Пример:
    Введите год: 2024
    2024 - високосный год
  \end{verbatim}

  \textit{\textbf{Задание 7.}}  
  Написать программу, которая запрашивает у пользователя три числа и выводит на экран наименьшее из них. При решении нельзя использовать встроенные функции min() и max().

  \begin{verbatim}
  Пример:
    Введите три числа: 8 3 5
    Наименьшее число: 3
  \end{verbatim}

  \textit{\textbf{Задание 8.}}  
  В магазине проводится акция. Акция работает по следующим правилам:

  \begin{vvsu_itemize}
    \item Сумма < 1000 => скидка - 0\%
    \item Сумма < 5000 => скидка - 5\%
    \item Сумма < 10000 => скидка - 10\%
    \item Сумма > 10000 => скидка - 15\%
  \end{vvsu_itemize}

  Напишите программу, которая запрашивает сумму покупки и выводит размер скидки и итоговую сумму к оплате.
  
  \begin{verbatim}
  Пример:
    Введите сумму покупки: 7500
    Ваша скидка: 10%
    К оплате : 6750.0
  \end{verbatim}
  
  \textit{\textbf{Задание 9.}}  
  Написать программу, которая определяет время суток по введенному часу (целое число от 0 до 23).

  \begin{vvsu_itemize}
    \item С 0 до 5 часов - ночь
    \item С 6 до 11 часов - утро
    \item С 12 до 17 часов - день
    \item С 18 до 23 часов - вечер
  \end{vvsu_itemize}
  
  \begin{verbatim}
  Пример:
    Введите час (0–23): 20
    Сейчас вечер
  \end{verbatim}

  \textit{\textbf{Задание 10.}}  
  Написать программу, которая определяет, является ли введенное число простым. Число называется простым, если оно больше 1 и делится только на 1 и само себя. Программа должна корректно обрабатывать некорректный ввод и выводить соответствующие сообщения об ошибках.
  
  \begin{verbatim}
  Пример:
    Введите число: 17
    17 - простое число
  \end{verbatim}
\end{addition}


% Содержание
\toc

% Глава - Выполнение работы
\section{Выполнение работы}

% Подглава - Задание 1
\subsection{Задание 1}

Сначала обозначим переменную temperature типа float. Далее просим пользователя ввести значение температуры, которое записываем в нашу переменную. После чего обрабатываем значение по условиям, если temperature >= 20, то выводим в консоль, кондиционер выключен, в ином случае, выведем, что он включен. На рисунке \ref{fig:code_task_1} представлен код полученной программы.

\begin{vvsu_figure}{Листинг программы для задания 1}{fig:code_task_1}
  \begin{minipage}{.75\textwidth}
    \lstinputlisting[language=C,basicstyle=\fontsize{10}{10}\linespread{1}\selectfont\ttfamily]{code/task1.c}
  \end{minipage}
\end{vvsu_figure}

% Подглава - Задание 2
\subsection{Задание 2}

Сначала заводим переменную monthNum типа integer. Далее просим пользователя ввести интересующий его месяц. Затем проверяем данные на корректность по условию 1 <= monthNum <= 12. И наконец обрабатываем по условиям. Если 3 <= monthNum <= 5, то выводим весна. Если 6 <= monthNum <= 8, то выводим лето. Если 6 <= monthNum <= 8, то выводим осень. А на оставшиеся варианты выводим зима. На рисунке \ref{fig:code_task_2} представлен код программы.

\begin{vvsu_figure}{Листинг программы для задания 2}{fig:code_task_2}
  \begin{minipage}{.75\textwidth}
    \lstinputlisting[language=C,basicstyle=\fontsize{10}{10}\linespread{1}\selectfont\ttfamily]{code/task2.c}
  \end{minipage}
\end{vvsu_figure}

% Подглава - Задание 3
\subsection{Задание 3}

Сначала заводим целочисленную переменную dogAge. Далее просим ввести возраст собаки и проверяем введеные данные на корректность по условию 1 <=dogAge <= 22. Затем конвертируем возраст собаки в человеческий по следующему алгоритму: если dogAge <= 2, то dogAge * 10.5, в ином случае мы получаем следующее выражение - 21 + (dogAge-2)*4. После выводим в консоль результат. На рисунке \ref{fig:code_task_3} представлен код программы.
\begin{vvsu_figure}{Листинг программы для задания 3}{fig:code_task_3}
  \begin{minipage}{.75\textwidth}
    \lstinputlisting[language=C,basicstyle=\fontsize{10}{10}\linespread{1}\selectfont\ttfamily]{code/task3.c}
  \end{minipage}
\end{vvsu_figure}

% Подглава - Задание 4
\subsection{Задание 4}

Сначала заведем перемнную num типа integer. Далее просим пользователя ввести интересющее его число. Затем проверим число на четность, используя оперцию побитового умножения, проверяя последний бит числа в двоичной системе счисления на наличие единицы - в случае ее наличия, число нечетное. После проверяем делится ли сумма цифр числа на 3 без остатка. В случае соблюдения всех условий, выводим в консоль - число делится на 6, в ином случае - не делится. На рисунке \ref{fig:code_task_4} представлен код решения.

\begin{vvsu_figure}{Листинг программы для задания 4}{fig:code_task_4}
  \begin{minipage}{.75\textwidth}
    \lstinputlisting[language=C,basicstyle=\fontsize{10}{10}\linespread{1}\selectfont\ttfamily]{code/task4.c}
  \end{minipage}
\end{vvsu_figure}


% Подглава - Задание 5
\subsection{Задание 5}

Для пешения нам понадобятся следующие переменные:
\begin{vvsu_itemize}
  \item массив password для типа char, в нем мы будем хранить пароль.
  \item literal типа integer, ее мы будем использовать для временного хранения числового представления обрабатываемого символа.
  \item i типа integer, ее будем использовать в качестве итератора, для перемещения по массиву из строк в цикле while.
  \item haveUpCase типа integer, будем исполльзовать в качестве логической. Присваиваем 1, если в пароле есть символы верхнего регистра.
  \item haveLowCase типа integer, будем исполльзовать в качестве логической. Присваиваем 1, если в пароле есть символы нижнего регистра.
  \item haveNum типа integer, будем исполльзовать в качестве логической. Присваиваем 1, если в пароле есть цифры.
  \item haveSpecLit типа integer, будем исполльзовать в качестве логической. Присваиваем 1, если в пароле есть специальные символы.
\end{vvsu_itemize}

Сначала просим пользователя ввести придуманный пароль, записывем его в password. Далее проверяем его длинну, если она более или равна 8 символам, то продолжаем, в ином случае сообщаем о ненадежности пароля. После в цикле while пробегаемся по символам пароля, проверяя каждый и отмечая с помощью логических перемнных наличие: символов верхнего и нижнего регистров, специальных символов и цифр. Символы обрабатываем по их числовому представлению, согласно базовой таблице ASCII. Когда символы закончились проверяем, что у нас есть все нобходимые символы, если они есть, от выводим в консоль, что пароль надежный, в ином случае говорим, что пароль не надежный и с помощью логических переменных определяем почему. На рисунке \ref{fig:code_task_5} представлен код программы.

\begin{vvsu_figure}{Листинг программы для задания 5}{fig:code_task_5}
  \begin{minipage}{.75\textwidth}
    \lstinputlisting[language=C,basicstyle=\fontsize{10}{10}\linespread{1}\selectfont\ttfamily]{code/task5.c}
  \end{minipage}
\end{vvsu_figure}

% Подглава - Задание 6
\subsection{Задание 6}

Сначала заводим переменную year типа integer. Затем просим пользователя ввести год и записываем в year, проверяя на корректность по условию - year >= 0. Далее обрабатываем значение, если оно делится одновременно на 4 и 100 без остатка или на 400, то выводим в консоль, что год високосный, в ином случае - не високосный. На рисунке \ref{fig:code_task_6} представлен код программы.

\begin{vvsu_figure}{Листинг программы для задания 6}{fig:code_task_6}
  \begin{minipage}{.75\textwidth}
    \lstinputlisting[language=C,basicstyle=\fontsize{10}{10}\linespread{1}\selectfont\ttfamily]{code/task6.c}
  \end{minipage}
\end{vvsu_figure}

% Подглава - Задание 7
\subsection{Задание 7}

Сначала зеведем массив для 3 целочисленных чисел и переменную maxNum типа integer и присвоим ей наименьшее значение, допустимое типом данных. Получаем от пользователя 3 числа, которые записываем в массив. Далее с помощью цикла бегаем по массиву и ищем самое большое число сравнивая с maxNum. После выводим в консоль наибольшее значение.На рисунке \ref{fig:code_task_7} представлен код программы.

\begin{vvsu_figure}{Листинг программы для задания 7}{fig:code_task_7}
  \begin{minipage}{.75\textwidth}
    \lstinputlisting[language=C,basicstyle=\fontsize{10}{10}\linespread{1}\selectfont\ttfamily]{code/task7.c}
  \end{minipage}
\end{vvsu_figure}

% Подглава - Задание 8
\subsection{Задание 8}

Сначала заведем переменные sum и discount типа float. Далее получаем сумму покупки от пользователя и записываем в sum. После проверяем корректность данных, сумма должна быть >= 0. Затем задаем условия, в которых описываем при каких условиях, какая скидка.
\begin{vvsu_itemize}
  \item Если sum < 1000, то dicount = 0.
  \item Если 1000 <= sum < 5000, то dicount = 0.05
  \item Если 5000 <= sum < 10000, то dicount = 0.1
  \item Во всех остальных случаях dicount = 0.15
\end{vvsu_itemize}
После выводим скидку и сумму к оплате, после применения скидки. На рисунке \ref{fig:code_task_8} представлен код программы.

\begin{vvsu_figure}{Листинг программы для задания 8}{fig:code_task_8}
  \begin{minipage}{.75\textwidth}
    \lstinputlisting[language=C,basicstyle=\fontsize{10}{10}\linespread{1}\selectfont\ttfamily]{code/task8.c}
  \end{minipage}
\end{vvsu_figure}

% Подглава - Задание 9
\subsection{Задание 9}

Сначала обозначим переменную hours типа integer. После просим пользователя ввести интересующий его час суток, проверяя на корректность по условию - 0 <= hours <= 23. Далее определяем условия по которым определяем текущее время суток:
\begin{vvsu_itemize}
  \item Если 0 <= hours <= 5, то выводим в консоль, что сейчас ночь.
  \item Если 6 <= hours <= 11, то выводим в консоль, что сейчас утро.
  \item Если 12 <= hours <= 17, то выводим в консоль, что сейчас день.
  \item Во всех остальных случаях выводим в консоль, что сейчас вечер.
\end{vvsu_itemize}  
На рисунке \ref{fig:code_task_9} представлен код программы.

\begin{vvsu_figure}{Листинг программы для задания 9}{fig:code_task_9}
  \begin{minipage}{.75\textwidth}
    \lstinputlisting[language=C,basicstyle=\fontsize{10}{10}\linespread{1}\selectfont\ttfamily]{code/task9.c}
  \end{minipage}
\end{vvsu_figure}

% Подглава - Задание 10
\subsection{Задание 10}

Заводим переменные num и numOfDeviders, инициализируемую 0. Далее просим пользователя ввести число, которое он хочет проверить на простоту, и запишем в num. После проверим, что num > 0. Затем, чтобы уменьшить дальнейшее кол-во итераций, проверим на четность. Поскольку любое четное число, кроме двойки, уже имеют минимум 4 делителя - 1, 2, само число и результат деления на 2. Дальше в цикле for пробегаем по числам от 1 до num и проверяем делимость без остатка, в случае наличия остатка к numOfDeviders прибавляем 1. Если numOfDeviders > 2, то выводим, что число составное, в ином случае выводим - простое. 
На рисунке \ref{fig:code_task_10} представлен код программы.

\begin{vvsu_figure}{Листинг программы для задания 10}{fig:code_task_10}
  \begin{minipage}{.75\textwidth}
    \lstinputlisting[language=C,basicstyle=\fontsize{10}{10}\linespread{1}\selectfont\ttfamily]{code/task10.c}
  \end{minipage}
\end{vvsu_figure}

\end{document}